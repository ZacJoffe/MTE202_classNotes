\chapter*{Lecture 25}
\begin{recall}{}{}
\begin{itemize}
\item Forced Vibration
\end{itemize}
\end{recall}




\subsubsection{Solving force-vibrations ODEs}
\begin{itemize}
\item[Step 1] The ODE is: 2nd-order, constant coefficient, non-homogeneous (since we have a forcing term).

Important to select the appropriate solution method:
\begin{itemize}
\item Reduction of order
\item Auxiliairy equations
\item Method of undetermined coefficients
\item Variation of parameters 
\end{itemize}
Since we have a non-homogeneous equation, we general select between:
\begin{itemize}
\item Method of undetermined coefficients if: $F(t)$ has the form of an exponential, sin/cos, polynomial, or combination of the above functions.
\item Variation of parameters: if the homogeneous solution is known. In other words, if the homogeneous solution (free vibration case is known).
\end{itemize}
\item[Step 2] Solve for the homogeneous case (if it is not already known). Identify if the solution is:
\begin{itemize}
\item over-damped
\item critically damped
\item under-damped
\end{itemize}
\item[Step 3] Solve the non-homogeneous equations with $F(t)$.
\end{itemize}




\begin{exmp}{Forced-vibrations:}\\
Solve a spring-mass-damper system with $F(t)=F_0\cos(\omega t)$ where $F_0$ and $\omega$ are constants.

\textbf{Solution:}
We solve the problem using MUC.\\
By looking up in the table, we find:
\begin{itemize}
\item $x_p = A\sin(\omega t)+B\cos(\omega t)$
\item $x'_p = A\omega\cos(\omega t)-B\omega\sin(\omega t)$
\item $x''_p = -A\omega^2\sin(\omega t)-B\omega^2\cos(\omega t)$\\
$x''_p=-\omega^2\left[A\sin(\omega t)+B\cos(\omega t)\right]$\\
$x''_p=-\omega^2x_p$\\
\end{itemize}
Sub into the ODE:
\begin{equation*}
m\left(-\omega^2x_p\right)+c\left(A\omega\cos(\omega t)-B\omega\sin(\omega t)\right) +kx_p = F_0\cos(\omega t)
\end{equation*}
We rearrange:
\begin{align*}
&\left(k-m\omega^2\right)\left[A\sin(\omega t)+B\cos(\omega t)\right]+c\omega\left(A\cos(\omega t)-B\sin(\omega t)\right) = F_0\cos(\omega t)\\
&\left(k-m\omega^2\right)x_p+c\omega\left(A\cos(\omega t)-B\sin(\omega t)\right)  = F_0\cos(\omega t)
\end{align*}
Sort by sine/cos:
\begin{align*}
&\left[A\left(k-m\omega^2\right)-B c\omega\right]\sin(\omega t) \\
&\qquad +\left[B\left(k-m\omega^2\right)+Ac\omega\right]\cos(\omega t)  = F_0\cos(\omega t)
\end{align*}

We have the following two equations:
\begin{itemize}
\item $A\left(k-m\omega^2\right)-B c\omega =0$
\item $B\left(k-m\omega^2\right)+Ac\omega=F_0$
\end{itemize}
Solving the equations, we obtain:
\begin{align*}
&A=\frac{F_0 c\omega}{(k-m\omega^2)^2+c^2\omega^2}\\
&B=\frac{F_0 (k-m\omega^2)}{(k-m\omega^2)^2+c^2\omega^2}
\end{align*}

We can sub the constants into the proposed solution and rearranging:
\begin{equation*}
x_p=\frac{F_0 }{(k-m\omega^2)^2+c^2\omega^2}\left[c\omega \sin(\omega t)+(k-m\omega^2)\cos(\omega t)\right]
\end{equation*}
Can be re-written into the form $Esin(\omega t+\phi)$. Recall:
\begin{align*}
x_p=\underbrace{E\sin(\phi)}_{c_1} \cos(\omega t)+\underbrace{E\cos(\phi)}_{c_2} \sin(\omega t)
\end{align*}
We can write:
\begin{itemize}
\item $E\sin(\phi)=(k-m\omega^2)$
\item $Ecos(\phi)=c\omega$
\end{itemize}
two equations, two unknowns ($E$ and $\phi$)

Where $\phi=arctan\left(\frac{k-m\omega^2}{c \omega}\right)$ and $E=\sqrt{c^2\omega^2+(k-m\omega^2)^2}$\\
The solution can take the form:
\begin{equation*}
x_p=\frac{F_0 }{\sqrt{(k-m\omega^2)^2+c^2\omega^2}}\sin\left(\omega t + \phi \right)
\end{equation*}
where $\omega$ is the natural frequency, $\phi$ the phase angle and the coefficient in front of the sine the amplitude.
\end{exmp}



Let's suppose that the system has no damping.
\begin{equation*}
x_p=\frac{F_0 }{{(k-m\omega^2)}}\sin\left(\omega t + \phi \right)=\frac{F_0 }{{m(\frac{k}{m}-\omega^2)}}\sin\left(\omega t + \phi \right)=\frac{F_0 }{{m(\omega^2_0-\omega^2)}}\sin\left(\omega t + \phi \right)
\end{equation*}

where $\omega_0/2\pi = \frac{\sqrt{k}}{2\sqrt{m}\pi}$ is the natural frequency of the system and $\omega$ forced frequency of the system. When both frequencies match, we have \textbf{resonance} of the system.


%
%\subsection{Gaussian elimination}
%For a system of 1st order ODEs:
%\begin{align*}
%a_1x'+a_2x+a_3y'+a_4y=f_1\\
%a_5x'+a_6x+a_7y'+a_8y=f_2
%\end{align*}
%We introduce the differential operator:
%\begin{align*}
%D=\frac{d}{dt}\qquad D^2=\frac{d^2}{dt^2}\\
%e.g. D(y) = \frac{dy}{dt} \qquad or D \cdot D(y) = \frac{d}{dt}\frac{dy}{dt}=\frac{d^2y}{dt^2}=D^2(y)
%\end{align*}
%We can rewrite the equation with the operators:
%\begin{align*}
%a_1D(x)+a_2x+a_3D(y)+a_4y=f_1\\
%a_5D(x)+a_6x+a_7D(y)+a_8y=f_2
%\end{align*}
%or:
%
%\begin{align*}
%(a_1D+a_2)[x]+(a_3D+a_4)[y]=f_1\\
%(a_5D+a_6)[x]+(a_7D+a_8)[y]=f_2
%\end{align*}
%The above equation can be solved by elimination!
%
%
%\begin{exmp}{Gaussian elimination: system of ODEs}\\
%Consider:
%\begin{align*}
%x'-3x+4y=1\\
%y'-4x+7y=10t
%\end{align*}
%where $x$ and $y$ are dependent variables and $t$ is the independent variable.
%
%\textbf{Solution:}\\
%Rewrite the equation with operator notation:
%\begin{align*}
%(D-3)[x]+4[y]=1\\
%-4[x]+(D+7)[y]=10[t]
%\end{align*}
%Use elimination technique to get rid of $x$:\\
%(1) x 4+ (2)x$(D-3)$\\
%\begin{align*}
%16 [y] +(D+7)(D-3)[y]=4+(D-3)10[t]
%\end{align*}
%We evaluate the RHS:
%\begin{align*}
%4+(D-3)10[t]=4+(\frac{d}{dt}-3)10[t]=4+(\frac{d[10[t]]}{dt}-30[t])=14-30[t]
%\end{align*}
%We evaluate the LHS:
%\begin{align*}
%16 [y] +(D+7)(D-3)[y]=16 [y] +(D^2+4D-21)[y]=(D^2+4D-5)[y]=\frac{d^2y}{dt^2}+4\frac{dy}{dt}-5y
%\end{align*}
%We combine both equations:
%\begin{align*}
%\boxed{\frac{d^2y}{dt^2}+4\frac{dy}{dt}-5y=14-30[t]}
%\end{align*}
%Now we have a second-order ODE for $y$ which can be solved via MUC!\\
%General solution of $y(t)$:
%\begin{align*}
%y_g=C_1e^{-5t}+c_2e^t+6t+2
%\end{align*}
%Now we can solve for $x$. From this equation:
%\begin{align*}
%-4[x]+(D+7)[y]=10[t]\\
%4[x]=\frac{d}{dt}y-7y-10[t]
%\end{align*}
%
%From the general solution of $y$, we know the derivative.


%\end{exmp}

