\chapter*{Lecture 4}
\begin{recall}{}{}
\begin{itemize}
\item Order of the equation tells us the number of necessary ICs
\item Initial Value Problem (IVP)
\item Solution methods for first-order ODEs
\end{itemize}


\end{recall}


\section{Substitution and Transformation (corresponds to 2.6 in book)} 
If the equation is not separable, we may still be able to solve it by applying a substitution and transformation.\\

Main idea: make the ODE separable!\\

Substitution procedure:
\begin{enumerate}
\item Identify the type of equation and determine the substitution or transformation (some level of intuition required)
\item Rewrite the original equation in terms of the new variable
\item Solve the transformed equation (e.g. direct integration)
\item Express the solution in terms of the original variables
\end{enumerate}

\subsection{Homogeneous equation}
Before, we continue, let's consider this equation:
\begin{equation}
\frac{dy}{dx}=f(x,y)
\end{equation}
Is this equation homogeneous? [We note here that "homogeneous equation" differs from our previous use of the term which was in the context of a "homogeneous linear differential equation"]. If the function $f(x,y)$ can be expressed as a ratio of $y/x$ (or $x/y$) the equation is homogeneous.  Typically, a homogeneous equation depends on the form of $f(x,y)$. For example, if $f(x,y)=xy$, the equation would be homogeneous. But, if we have $f(x,y)=x+y$, we have a non-homogeneous equation. But what about this $f(x,y)=\frac{x+y}{x}$?\\

Here we show that through substitution, we can obtain a homogeneous for the last example. More generally, if the RHS can be written as a ratio of $y/x$ alone, then we say that the equation is \textbf{homogeneous}.

If we can write $f(x,y)=f(y/x)$, we can substitute a new variable $v=y/x$ such that the RHS is $f(v)$. We have:
\begin{equation}
\frac{dy}{dx}=f(v)
\label{midTransform}
\end{equation}
Now we want to also transform the LHS, we see that $y=vx$. By deriving this equation with respect to $x$, we obtain:
\begin{equation}
\frac{dy}{dx}=x\frac{dv}{dx}+v
\end{equation}
By replacing the above relation in \eqref{midTransform}, we obtain:
\begin{equation}
x\frac{dv}{dx}+v = f(v)
\end{equation}

or 
\begin{equation}
\frac{dv}{dx}+\frac{v- f(v)}{x}=0
\end{equation}
which is a homogeneous equation!


Let's show these concepts through examples.
\begin{center}
\noindent\rule{4cm}{0.4pt}
\end{center}

\begin{exmp}{Substitution and transformation.}
Solve the following ODE:
\begin{equation}
\frac{dy}{dx}=\left(\frac{x}{y}\right)^2+\frac{y}{x}
\end{equation}
Solution: It is clear that the RHS can be written as $F\left(\frac{x}{y}\right)$ or $F\left(\frac{y}{x}\right)$, therefore we can solve the equation by direct integration after a substitution and transformation.
\begin{enumerate}
\item Change the dependent variable: let $y/x=v$ or $y=xv$. Now it is important to recall that $v$ is a function of $x$, therefore, when differentiating the substitution with respect to $y$, we have:
\begin{equation}
\boxed{\frac{dy}{dx}=x\frac{dv}{dx}+v}
\end{equation}
\item  Substitute the above transformation into the original ODE:
\begin{equation}
x\frac{dv}{dx}+v = \left(\frac{x}{y}\right)^2+\frac{y}{x} = \left(\frac{1}{v}\right)^2+v = f(v)
\end{equation}
 and rearrange:
\begin{equation}
\frac{dv}{dx} = \underbrace{\left[f(v)-v\right]}_{G(v)}\underbrace{\frac{1}{x}}_{H(x)}
\end{equation}
 Separable equation!
 
 \item Separate and integrate (direct integration)
\begin{equation}
\int \frac{dv}{f(v)-v} =\int \frac{1}{x} dx
\end{equation}
 The first term is simplified to:
\begin{equation}
 \int \frac{1}{f(v)-v}dv = \int \frac{1}{\left(\frac{1}{v^2}+v\right)-v}dv= \int v^2 dv=\frac{v^3}{3}+C_1
\end{equation}
\begin{equation}
\int \frac{1}{x} dx = ln\left|x\right|+C_2
\end{equation}

\item Substitute $v=y/x$ back into the equation
\begin{equation}
 \frac{v^3}{3}=\ln\left|x\right| +C = \frac{\left(\frac{y}{x}\right)^3}{3}=\ln\left|x\right|+C
\end{equation}

The solution is:
\begin{equation}
 y=x \left(3\ln\left|x\right|+D\right)^{1/3} 
\end{equation}
\end{enumerate}
\end{exmp}


\begin{center}
\noindent\rule{4cm}{0.4pt}
\end{center}

\begin{exmp}{Substitution and transformation:}
Solve the following ODE.
\begin{equation*}
\frac{dy}{dx}=\frac{xy+2y^2}{x^2}
\end{equation*}
Assuming the initial condition: $y(1)=\frac{1}{2}$.\\

Note: This equation can be made separable as the RHS can be written as $F(y/x)=F(v)$, where $v=y/x$.
\begin{equation*}
\frac{dy}{dx}=\frac{y}{x}+2\left(\frac{y}{x}\right)^2= v+2v^2
\end{equation*}

\textbf{Solution:}
\begin{enumerate}
\item Let $v=y/x$, therefore: $\frac{dy}{dx}=x\frac{dv}{dx}+v$
\item Substitute into the ODE:
\begin{equation*}
\frac{dy}{dx}=x\frac{dv}{dx}+v=v+2v^2
\end{equation*}
after simplification:
\begin{equation*}
\frac{dv}{dx}=2\frac{v^2}{x}
\end{equation*}

\item Separate and integrate:
\begin{equation*}
\int \frac{dv}{v^2}=\int \frac{2}{x}dx
\end{equation*}
\begin{equation*}
-\frac{1}{v}=2 \ln\left|x\right| +C
\end{equation*}
or
\begin{equation*}
v=\frac{-1}{2 \ln\left|x\right|+C}
\end{equation*}
\item Replace original variables: $y=vx$. The general solution is:
\begin{equation*}
y=\frac{-x}{2 \ln\left|x\right|+C_2}
\end{equation*}
\item Find particular solution. To find the value of $C$, apply the condition at $x=1$ then $y=1/2$, therefore $C=-2$.\\
Final solution is:

\begin{equation*}
\boxed{y=\frac{-x}{2 \ln\left|x\right|-2}}\qquad \bLozenge
\end{equation*}

\end{enumerate}
\begin{center}
\noindent\rule{4cm}{0.4pt}
\end{center}
\end{exmp}


\begin{exmp}{Other substitutions:}
Try to solve this simple ODE:
\begin{equation}
\left(2x-4y+5\right)y' +x-2y+3=0
\end{equation}
\textbf{Solution:}
There is no clear cut answer about what substitution we can use. With a bit of intuition, we propose to use $v=x-2y$. From this transformation, we can compute the derivative of $y$ with respect to $x$:
\begin{equation*}
y'=(1-v')/2
\end{equation*}
By substitution, we find:
\begin{equation}
\left(2v+5\right)((1-v')/2) +v+3=0
\end{equation}
or
\begin{equation}
\left(2v+5\right)v' =4v+11
\end{equation}
We can now separate the above equation (multiply by 2 and divide by RHS):
\begin{equation}
\frac{\left(2v+5\right)}{\left(4v+11\right)}dv  =\frac{4v+11-1}{4v+11}dv=\left(1-\frac{1}{4v+11}\right)dv =2dx
\end{equation}
By integration:
\begin{equation}
v-\frac{1}{4}\ln\left|4v+11\right|=2x+c
\end{equation}
Now we can transform back into our original variables ($v=x-2y$) to find our implicit solution:
\begin{equation*}
\boxed{4x+8y+\ln\left|4x-8y+11\right|=c} \qquad \bLozenge
\end{equation*}

\end{exmp}

\subsection{Bernouilli equation }
If a first-order equation can be written as:
\begin{equation}
\frac{dy}{dx}+P(x)y=Q(x)y^n
\end{equation}
(assuming that $P(x)$ and $Q(x)$ are continuous over the interval of interest). We can solve this equation via substitution and transformation using the following substitution:
\begin{equation}
v=y^{1-n}
\end{equation}
Note that if $n=0$ or 1, we have a linear equation that is more easily solved using the linear equation method (seen a little later).

\updateinfo[September 19, 2018]