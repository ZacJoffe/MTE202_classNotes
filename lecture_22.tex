\chapter*{Lecture 22}
\begin{recall}{}{}
\begin{itemize}
\item MUC the judicious guess.
\item MUC superposition
\item MUC double root case
\item  $F(x)$ is limited to simple cases (which maintain the same general form under differentiation, e.g. polynomial, exponents, sines)
\end{itemize}
\end{recall}


Ok, time to pull your hair out a bit...
\begin{exmp}{MUC (special case):}\\
Solve:
\begin{equation*}
y''+y=\cos(x)
\end{equation*}
Solution:
\begin{itemize}
\item Find the homogeneous solution:\\
\begin{equation*}
r^2+1=0 \qquad r=\pm i \qquad \alpha=0; \beta=1
\end{equation*}
Two complex roots!
\begin{equation*}
\boxed{y_h=C_1 \cos(x)+C_2 \sin(x)}
\end{equation*}
\item Particular solution: Since $F(x)=\cos(x)$ we assume $y_p=a\cos(x)+b\sin(x)$.
\begin{align*}
y'_p=-a\sin(x)+b\cos(x)\\
y''_p=-a\cos(x)-b\sin(x)\\
\end{align*}
Sub into the ODE:
\begin{align*}
(-a\cos(x)-b\sin(x))+a\cos(x)+b\sin(x)=\cos(x)\\
0 = \cos(x)
\end{align*}
Wrong! Cannot determine $a$ and $b$ for $y_p$!!!!
We made a wrong guess!
If we check the homogeneous solution, it takes the same standard form as the particular solution. Using that info, we try a new guess: $y_p=ax\cos(x)+bx\sin(x)$.
\begin{align*}
y'_p=(a+bx)\cos(x)+(b-ax)\sin(x)\\
y''_p=(2b-ax)\cos(x)-(2a+bx)\sin(x)
\end{align*}
sub into the ODE:
\begin{align*}
\underbrace{(2b-ax)\cos(x)-(2a+bx)\sin(x)}_{y''_p}+\underbrace{ax\cos(x)+bx\sin(x)}_{y_p}=\cos(x)
\end{align*}
After simplifications:
\begin{align*}
2b\cos(x)-2a\sin(x)=\cos(x)\\
2b=1; \text{or } b=1/2; \qquad 2a=0 \text{or } a=0
\end{align*}
The particular solution is then:
\begin{align*}
y_p=1/2x \sin(x)
\end{align*}
\item General solution:
\begin{align*}
\boxed{y_g=C_1 \cos(x)+C_2 \sin(x)+\frac{1}{2}x \sin(x)}
\end{align*}

\end{itemize}
\end{exmp}
\textbf{Conclusion:} If the guessed $y_p$ is the same as $y_h$, try $y_p=xy_c$!





\section{Variation of Parameter - VoP}
This approach can be applied to 2nd order, linear, non-homogeneous ODEs with \textbf{variable} coefficients and for arbitrary RHS: $F(x)$.

We write our second-order ODE in general form:
\begin {equation*}
y''+p(x) y' + q(x)y=F(x)
\end {equation*}

The general solution of the \textbf{homogeneous} equation is:
\begin {equation*}
y_h=c_1 y_1(x) +c_2 y_2(x)
\end {equation*}
Here $c_1$ and $c_2$ can be considered parameters of the ODE (thus the name of the method). The method of variation of parameters involves replacing the parameters with functions $v_1(x)$  and  $v_2(x)$ so that we obtain a particular solution of the form:
\begin {equation*}
y_p=v_1(x) y_1(x) +v_2(x) y_2(x)
\end {equation*}

By differentiating, we obtain:
\begin {align*}
y'_p=v'_1(x) y_1(x)+v_1(x) y'_1(x) +v'_2(x) y_2(x)+v_2(x) y'_2(x)\\
y'_p=\left(v'_1(x) y_1(x)+v'_2(x) y_2(x)\right)+\left(v_1(x) y'_1(x)+v_2(x) y'_2(x)\right)
\end {align*}
We have added two new functions $v_1$ and $v_2$. We know that $y_1$ and $y_2$ are constrained being solutions of the homogeneous ODE. Currently, we have only imposed one constraint on the new functions, namely that $y_p$ must satisfy the non-homogeneous ODE. We can impose a second constraint. Here for simplicity, we state that:
\begin {align*}
\left(v'_1(x) y_1(x)+v'_2(x) y_2(x)\right)=0
\end {align*}
Therefore, we obtain:
\begin {align*}
y'_p=\left(v_1(x) y'_1(x)+v_2(x) y'_2(x)\right)\\
y''_p=\left(v'_1(x) y'_1(x)+v'_2(x) y'_2(x)\right) +\left(v_1(x) y''_1(x)+v_2(x) y''_2(x)\right)
\end {align*}
Now, we can substitute these equations into the original ODE:
\begin {align*}
y_p''+p  y_p' + q y_p=F(x)\\
\left(v'_1  y'_1 +v'_2  y'_2  +v_1  y''_1 +v_2  y''_2 \right) +p \left(v_1  y'_1 +v_2  y'_2 \right)+q (v_1  y_1  +v_2  y_2) =F(x)\\
\end {align*}
We can now group the common term together:
\begin {align*}
\left(v'_1 y'_1 +v'_2 y'_2 \right) + v_1 \underbrace{\left(y''_1 +p y'_1+q y_1 \right)}_{=0} + v_2 \underbrace{\left( y''_2 +py'_2 + qy_2 \right)}_{=0}=F(x)
\end {align*}
Remember $y_1$ and $y_2$ are solutions to the HOMOGENEOUS equation!

Therefore, we obtain:
\begin {align*}
v'_1 y'_1 +v'_2 y'_2 =F(x)\\
v'_1 y_1 +v'_2 y_2 =0
\end {align*}
An algebraic manipulation of the equations lead to:
\begin {align*}
v'_1 =\frac{-F(x) y_2}{y_1y'_2-y'_1y_2} \qquad \text{and}\qquad v'_2 =\frac{F(x) y_1}{y_1y'_2-y'_1y_2}
\end {align*}
In what situation will the denominator be zero????
The denominator represents the Wronskian which is non-zero for linearly independent solution, therefore if $y_1$ and $y_2$ are solutions, they will be linearly independent!


The integration of the above equations yield:
\begin {align*}
v_1 =\int\frac{-F(x) y_2}{y_1y'_2-y'_1y_2} dx \qquad \text{and}\qquad v_2 =\int \frac{F(x) y_1}{y_1y'_2-y'_1y_2}dx
\end {align*}
What about integrative constants when plugging it back in to the ODE? 

\updateinfo[October 30, 2018]