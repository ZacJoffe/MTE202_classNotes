
\chapter*{Lecture 31}

\begin{recall}{}{}
\begin{itemize}
\item Inverse Laplace transform (partial fractions)
\end{itemize}
\end{recall}


\section*{Properties of the inverse LT}
The general approach for the method of partial fraction. Given:
\begin{align*}
F(s)=\frac{P(s)}{Q(s)}
\end{align*}
where $P(s)$ and $Q(s)$ are polynomials and the degree (highest power) of $P(s)<Q(s)$. In factoring $Q(s)$ many cases may arise:

\begin{enumerate}
\item Unrepeated linear factors
\item Repeated factors
\item Unrepeated irreducible quadratic factors
\end{enumerate}


\begin{enumerate}
\item $Q(s)$ has real, linear, non-repeated factors, i.e.:
\begin{align*}
Q(s)=(s-r_1)(s-r_2)\hdots (s-r_n)
\end{align*}
In this case, partial fractions can be directly used:
\begin{align*}
F(s)=\frac{A_1}{s-r_1}+\frac{A_2}{s-r_2}+\hdots+\frac{A_n}{s-r_n}
\end{align*}
where $A_i$ are real numbers.

\begin{exmp}{}
Solve the IVP:
\begin{equation*}
y''+y'-6y=1 \qquad \text{with }y(0)=0,y'(0)=1
\end{equation*}
\textbf{Solution:}\\
Using the derivative formulas, we obtain:
\begin{equation*}
(s^2+s-6)Y = 1+\frac{1}{s}=\frac{s+1}{s}
\end{equation*}
(note the slightly different notation for the Laplace transform)\\
We can reformulate $(s^2+s-6)=(s-2)(s+3)$. Since they are unrepeated factors, we can write:
\begin{equation*}
Y = \frac{s+1}{s(s-2)(s+3)}=\frac{A_1}{s}+\frac{A_2}{s-2}+\frac{A_3}{s+3}
\end{equation*}
Now we find the constants $A_1$,$A_2$ and $A_3$:
\begin{equation*}
{s+1}=(s-2)(s+3)A_1+s(s+3)A_2+s(s-2)A_3
\end{equation*}
We can do an  expansion or we can be cleaver and look at what happens at $s=0$, $s=2$ and $s=-3$:
\begin{align*}
1&=-2*3*A_1\\
3&=2*5A_2\\
-2=-3(-5)A_3
\end{align*}
Hence, we find
\begin{align*}
A_1=-1/6\\
A_2=3/10\\
A_3=-2/15
\end{align*}
The answer is:
\begin{equation*}
\Lapl^{-1}(Y)=-1/6+3/10e^{2t}-2/15e^{-3t}
\end{equation*}
\end{exmp}

%========================================
\item $Q(s)$ has repeated factors\\
Repeated factors e.g.: $(s-a)^2$ require partial fractions.
\begin{exmp}{}
Solve the IVP:
\begin{equation*}
y''-3y'+2y=4t \qquad \text{with }y(0)=1,y'(0)=-1
\end{equation*}
\textbf{Solution:}\\
We obtain the subsidiary equation:
\begin{equation*}
(s^2Y-s+1)-3(sY-1)+2Y=\frac{4}{s^2}
\end{equation*}
By collecting the terms, we have:
\begin{equation*}
(s^2-3s+2)Y=\frac{4+s^3-4s^2}{s^2}
\end{equation*}
Since $(s^2-3s+2)=(s-2)(s-1)$ and $s^2$ is a double factor, we have:
\begin{equation*}
Y=\frac{4+s^3-4s^2}{s^2(s-2)(s-1)}=\frac{A_2}{s^2}+\frac{A_1}{s}+\frac{B}{s-2}+\frac{C}{s-1}
\end{equation*} 
We multiply the RHS to get everything on the same denominator:
\begin{equation}
4+s^3-4s^2=A_2(s-2)(s-1)+A_1s(s-2)(s-1)+Bs^2(s-1)+Bs^2(s-2)
\label{eq:pfactor}
\end{equation} 
We find the zeroing factors. For $s=1$, we have: $C=-1$. For $s=2$, we have $B=-1$. Now for $s=0$, we get: $4=2A_2$, hence $A_2=2$.\\
By differentiating $\ref{eq:pfactor}$, we get:

\begin{equation}
3s^2-8s=A_2(2s-3)+A_1(s-2)(s-1)+\text{ other terms}
\end{equation} 
Now, for $s=0$, we find $A_1=3A_2/2=3$
The solution becomes:

\begin{equation*}
y=\Lapl^{-1}(Y)=2 t+3-2e^{2t}-e^{t}
\end{equation*} 


\end{exmp}



\end{enumerate}





