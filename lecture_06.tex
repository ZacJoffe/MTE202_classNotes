\chapter*{Lecture 6}
\begin{recall}{}{}
\begin{itemize}
\item Exact Equations
\end{itemize}
\end{recall}

\begin{exmp}{}
Solve:
\begin{equation*}
\frac{dy}{dx}=-\frac{y\cos(x)+2xe^{y}}{\sin(x)+x^{2}e^{y}-1}
\end{equation*}
Exact equation?
\begin{enumerate}
\item Reformulate
\begin{equation*}
\underbrace{(y\cos(x)+2xe^{y})}_{M(x,y)}{dx}+\underbrace{(\sin(x)+x^{2}e^{y}-1)}_{N(x,y)}dy=0
\end{equation*}
\item  check the compatibility equation:
\begin{equation*}
\frac{\partial M(x,y)}{\partial y}=(\cos(x)+2xe^{y}) \qquad \text{and}\qquad \frac{\partial N(x,y)}{\partial x}=(\cos(x)+2xe^{y}) 
\end{equation*}
It is Exact!
\item  we have a total derivative of a function $f(x,y)$:
\begin{equation*}
M(x,y) = \frac{\partial f(x,y)}{\partial x}=y\cos(x)+2xe^{y}
\end{equation*}

By integrating to find $f(x,y)$
\begin{equation*}
 \frac{\partial f(x,y)}{\partial x}=y\cos(x)+2xe^{y}
\end{equation*}

\begin{equation*}
 f(x,y)=\int (y\cos(x)+2xe^{y})dx+g(y)=y\sin(x)+x^2e^{y}+g(y)
\end{equation*}
\item find $g(y)$

\begin{equation*}
 \frac{\partial f(x,y)}{\partial y}=N=\sin(x)+x^2e^{y}+g'(y)=(\sin(x)+x^{2}e^{y}-1)
\end{equation*}
Therefore $g'(y)=-1$ and $g(y)=-y +c$
\end{enumerate}
\item Final solution is:

\begin{equation*}
 \boxed{f(x,y)=y\sin(x)+x^2e^{y}-y+c}
\end{equation*}
\end{exmp}
\begin{center}
\noindent\rule{4cm}{0.4pt}
\end{center}




\section{Integrating factor}
\begin{exmp}{}
Given the following DE:
\begin{equation}
\left(2y-6x\right)dx +\left(3x-4x^2y^{-1}\right)dy=0
\end{equation}
It is NOT exact. But, if we multiply the above equation by $xy^2$, it is transformed into an exact one!
\begin{equation}
xy^2\left(2y-6x\right)dx +xy^2\left(3x-4x^2y^{-1}\right)dy=0
\end{equation}
\label{simpleIntFactor}
\end{exmp}
(verify!)




Basic idea of integrating factors:  multiply a non-exact equation, e.g.:
\begin{equation}
P(x,y)dx+Q(x,y)dy=0
\end{equation}
By a function $U(x,y)$, such that the resulting equation:
\begin{equation}
U(x,y)P(x,y)dx+U(x,y)Q(x,y)dy=0
\end{equation}
is EXACT! 

We call $U(x,y)$ an \textbf{integrating factor}.

We find a function $U(x,y)$ such that (compatibility condition):
\begin{equation}
\frac{\partial U(x,y)P(x,y)}{\partial y}=\frac{\partial U(x,y)Q(x,y)}{\partial x}
\end{equation}



\subsection{How do we find integrating factors?}
For simple cases (such as \ref{simpleIntFactor}), the integrating factors may be found by analysis or inspection. But for most cases, we need a more formal approach to find the integrating factors.


We know that the integrating factor MUST satisfy the following condition:
\begin{equation}
\frac{\partial U(x,y)P(x,y)}{dy}=\frac{\partial U(x,y)Q(x,y)}{dx}
\label{chainrule}
\end{equation}

Therefore, we can apply the product rule to obtain:
\begin{equation}
P(x,y)\frac{\partial U(x,y)}{\partial y}+U(x,y)\frac{\partial P(x,y)}{\partial y}=Q(x,y)\frac{\partial U(x,y)}{dx}+U(x,y)\frac{\partial Q(x,y)}{\partial x}
\end{equation}

This equation is \textbf{more} complex...unless we assume that the integrating factor is only a function of ONE variable: e.g. $U=U(x)$. In which case: $\frac{\partial U(x,y)}{dy} =0$ and $\frac{\partial U(x,y)}{\partial x}=\frac{d U(x,y)}{dx}$. 

We obtain:

\begin{equation}
\frac{1}{U}\frac{dU}{dx}=\frac{1}{Q}\left(\frac{\partial P(x,y)}{\partial y}-\frac{\partial Q(x,y)}{\partial x}\right)
\end{equation}
By setting the RHS equal to $R$ yields:

\begin{equation}
\frac{1}{U}\frac{dU}{dx}=R
\end{equation}
or
\begin{equation}
\boxed{U(x)=e^{\int R(x) \, dx}}
\end{equation}
where we have omitted the constant of integration.



\begin{center}
\noindent\rule{4cm}{0.4pt}
\end{center}

\begin{exmp}{Integrating factors:}
Solve the following IVP
\begin{equation}
(e^x-\sin(y))dx+\cos(y)dy=0
\end{equation}

\textbf{Solution:}
\begin{enumerate}
\item Check for exactness: 
\begin{equation}
P=e^x-sin(y) \qquad \text{and} \qquad Q=\cos(y)
\end{equation}
and
\begin{equation}
\frac{\partial P}{\partial y}=-\cos(y)\qquad \text{and} \qquad \frac{\partial Q}{\partial x}=0
\end{equation}
the equation is NOT exact!
\item Find the integrating factors
\begin{equation}
R=\frac{1}{Q}\left(P_y-Q_x\right)=\frac{1}{\cos(y)}\left(-\cos(y)-0\right)=\frac{-cos(y)}{\cos(y)}=-1
\end{equation}
We compute the integrating factor as:
\begin{equation}
U(x)=\exp \int R(x)dx=\exp{\int -1 dx} =e^{-x}
\end{equation}
Our integrating factor is $e^{-x}$!

We can re-write the original equation as an exact equation!
\begin{equation}
\underbrace{e^{-x}(e^x-\sin(y))}_{M(x,y)}dx+\underbrace{e^{-x}\cos(y)}_{N(x,y)}dy=0
\end{equation}
(not a bad idea to double check)

\item Solve the exact equation: for an exact equation, we know that:
\begin{equation}
F(x,y)=\int M dx=  \int ((1-e^{-x}\sin(y))) dx+ g(y) = x+e^{-x}\sin(y)+g(y)
\end{equation}
From the above equation, we can now derive with respect to $y$ in order to find the term $g(y)$
\begin{equation}
\frac{\partial F(x,y)}{\partial y}= e^{-x}\cos(y)+g'(y) = \underbrace{e^{-x}\cos(y)}_{N(x,y)}
\end{equation}
Hence:  $g'(y)=0$ and $g(y)=cst$.
The general solution is then:
\begin{equation}
F(x,y)= x+e^{-x}\sin(y)+c
\end{equation}
\end{enumerate}
\end{exmp}
\begin{center}
\noindent\rule{4cm}{0.4pt}
\end{center}

\updateinfo[September 19, 2018]

