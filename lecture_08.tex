
\chapter*{Lecture 8}
\begin{recall}{}{}
\begin{itemize}

\item Integrating factors
\item Introduction to linear first-order equations
\end{itemize}
\end{recall}


\section{First-order Linear Equations (corresponds to 2.3 in book)}

Recall (Chap 1): a \textbf{linear first-order equation} is an equation that can be written in the form:
\begin{equation}
a_1(x)\frac{dy}{dx}+a_0(x) y=b(x)
\label{firstorderlinear}
\end{equation}

The above linear DE can be easily solved in the following three cases:
\begin{enumerate}
\item \textbf{If $a_0=0$}. In this case, we obtain:
\begin{equation*}
a_1(x)\frac{dy}{dx} =b(x)
\end{equation*}
which can be solved by direct integration. 
\item \textbf{If $b(x)=0$}. We have a homogeneous equation:
\begin{equation*}
\frac{dy}{dx} =-\frac{a_0(x)}{a_1(x)} y= F(x)G(y)
\end{equation*}
which is separable and can then solved via direct integration.
\item \textbf{If $a_0=a'_1$}. This is a bit less obvious. 
\begin{equation*}
a_1(x)\frac{dy}{dx}+\frac{da_1(x)}{dx} y=\frac{d\,a_1 y}{dx}=b(x)
\label{ODElin}
\end{equation*}
This product rule allows us to solve a much simpler problem!
\end{enumerate}



Recall:
\begin{equation*}
a_1(x)\frac{dy}{dx}+\underbrace{\frac{da_1(x)}{dx}}_{a_0} y=\frac{d\,a_1 y}{dx}=b(x)
\end{equation*}
We can easily solve the problem is $a_0=\frac{da_1(x)}{dx} $!! \\

\noindent\fbox{%
    \parbox{\textwidth}{The main idea of the linear equation solution is to reformulate the linear, first-order ODE \eqref{firstorderlinear} into a derivative of a coefficient times the dependent variable. In most cases we will resort to an "integrating factor".}} \par
\vspace{0.75cm}



We will see that ODEs are written in standard form:
\begin{equation}
\boxed{\frac{dy}{dx}+P(x) y=Q(x)}
\end{equation}
(note the abuse of notation with previous section with regards to $P$ and $Q$)\\
where $P(x)=a_0/a_1$ and $Q(x)=b/a_1$.\par

We seek to find an \textbf{integrating factor} such that :
\begin{equation*}
U(x)\frac{dy}{dx}+\underbrace{\textcolor{red}{U(x)P(x)}}_\text{same} y=U(x)Q(x) = \frac{d\, [U(x) y]}{dx} =U(x)\frac{dy}{dx}+\underbrace{\textcolor{red}{U'(x)}}_{same} y 
\end{equation*}
 We need to find $U(x)$ such that:
 \begin{equation}
 U' = \frac{d\, U}{d\,x}= UP
 \label{intFac}
 \end{equation}


The integrating factor can be found from \eqref{intFac}:
 \begin{equation*}
U(x) = e^{\int P(x) dx}
 \end{equation*}
(we can use any integration constant)\\

The correct integrating factor, $U(x)$, means that we can solve the following equation:
\begin{equation*}
\frac{d\,}{dx}\left[U(x) y\right]= U(x)Q(x)
\end{equation*}
The solution is then (trivially):
\begin{equation}
y(x)=\frac{1}{U(x)}\left[\int U(x)Q(x)dx + C \right]
\end{equation}
(IMPORTANT: The integrative constant is within the brackets!)


\textbf{Solution strategy:}
\begin{enumerate}
\item Rewrite in standard form:
\begin{equation*}
\frac{dy}{dx}+P(x) y=Q(x)
\end{equation*}
Make sure it is linear!!! 
\item Compute the integrating factor $U(x)$ such that:
 \begin{equation*}
 U' = UP \qquad \qquad \text{or}\qquad \qquad U(x)=e^{\int P(x) dx}
 \end{equation*}
\item Multiply the ODE by the integrating factor:
 \begin{equation*}
\frac{d\,}{dx} (U(x)y) = U(x)Q(x) 
\end{equation*}

\item Integrate and reorganize the last equation to obtain an explicit solution.
\end{enumerate}

\begin{center}
\noindent\rule{4cm}{0.4pt}
\end{center}


\begin{exmp}{Linear Differential Equations:}\\
Solve:
\begin{equation*}
y'-y=e^{2x}
\end{equation*}
\textbf{Solution:}\\
\begin{enumerate}
\item Write in standard for: (check)
\item Find integrating factor:
 \begin{equation*}
U(x) = e^{\int P(x) dx}= e^{\int(-1) dx}=e^{-x}
 \end{equation*}
 \item Multiply the ODE by the integrating factor
 \begin{equation*}
U(x)y'-U(x)y=\frac{d\, Uy}{dx}=U(x)e^{2x}=e^{-x}e^{2x}
\end{equation*}
\item Solve the above equation:
 \begin{equation*}
\frac{d\, Uy}{dx}=e^{-x}e^{2x}=e^x
\end{equation*}
 \begin{equation*}
y=\frac{1}{e^{-x}}\left[\int{e^x} dx\right]=e^{2x}+ce^x
\end{equation*}
\end{enumerate}

\end{exmp}

\begin{center}
\noindent\rule{4cm}{0.4pt}
\end{center} 

\begin{exmp}{Linear Differential Equations:}\\
Solve:
\begin{equation}
y'+y \tan(x) =\sin(2x) \qquad \qquad \text{with}\qquad y(0)=1
\end{equation}
\textbf{Solution:}\\
\begin{enumerate}
\item Write the equation in standard form:
\begin{equation*}
y'+\underbrace{\tan(x)}_{P(x)} y =\underbrace{\sin(2x)}_{Q(x)} \qquad \qquad \text{with}\qquad y(0)=1
\end{equation*}
\item Find the integrating factor:
 \begin{equation*}
U(x) = e^{\int tan(x) dx}
 \end{equation*}
 Recall: $\int tan(x) dx=\ln \left| \sec x\right|$, therefore:
  \begin{equation*}
U(x) = e^{\int tan(x) dx}=\left| \sec x \right |
 \end{equation*}
 \item Multiply the ODE by the integrating factor:
 \begin{equation*}
\frac{d\,}{dx} (U(x)y) = U(x)Q(x) = \sec x {\sin(2x)} 
\end{equation*}
\item Solve:
  \begin{equation*}
y = \frac{1}{U(x)} \int\sec x {\sin(2x)}dx =\frac{-2cos(x)}{sec(x)} + \frac{C}{sec(x)}= -2cos^2(x) + \frac{C}{sec(x)}
\end{equation*}
where $C=3$ (check IC).
The particular solution is:
  \begin{equation*}
y = -2cos^2(x) + 3cos(x)
\end{equation*}
\end{enumerate}
\end{exmp}
\begin{center}
\noindent\rule{4cm}{0.4pt}
\end{center}

\updateinfo[September 25, 2018]

