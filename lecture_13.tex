\chapter*{Lecture 13}
\begin{recall}{}{}
\begin{itemize}
\item Introduction to 2nd order ODEs
\end{itemize}
\end{recall}





\section{Solution superposition in homogeneous ODEs}
Let's suppose you have the following (homogeneous) second-order ODE:
\begin{equation*}
\frac{d^2 y}{dx^2}=y
\end{equation*}
By looking at the equation we can identify the solutions:
\begin{itemize}
\item $y=e^x$ since $\frac{d^2 y}{dx^2}= \frac{d^2 e^x}{dx^2}=e^x$
\item $y=e^{-x}$ since $\frac{d^2 y}{dx^2}= \frac{d^2 e^{-x}}{dx^2}=e^{-x}$
\end{itemize} 
what about this solution?
\begin{equation*}
y=c_1 e^{x}+ c_2e^{-x}
\end{equation*}
For the sake of the example, we suppose that $c_1=3$ and $c_2=2/5$.

\begin{eqnarray*}
y''-y = (c_1 e^{x}+ c_2e^{-x})''- (c_1 e^{x}+ c_2e^{-x})=\\
3 e^{x} + 2/5e^{-x} - (3 e^{x}+ 2/5e^{-x})=0
\end{eqnarray*}


This last solution tells us that there are an infinite number of solutions to this ODE (since the constants may take any values).
\\
\textbf{For a \emph{homogeneous linear equation}, we can always obtain new solutions from known solutions by multiplication by constant or by addition.} \\

This is called a \textbf{linear combination} of solutions. We will refer to this as the  \textbf{superposition principle} or the \textbf{linearity principle}.\\

NOTE: this theory does not hold for non-homogeneous linear equations nor for non-linear equations!!!

\begin{center}
\noindent\rule{4cm}{0.4pt}
\end{center}

\begin{exmp}{Linearity principle of a nonhomogeneous DE:}\\
Suppose the following ODE:
\begin{equation*}
y''+y=1
\end{equation*}
(this is a nonhomogeneous, linear, 2nd-order ODE).
Show that the priciple of linearity does NOT hold.\\
\textbf{Solution:}
The solutions of this ODE are $y_1=1+cos(x)$ and $y_2=1+sin(x)$. Is the linear superposition of these two also a solution to the ODE?\\
$y=c_1 y_1 + c_2 y_2 = c_1(1+cos(x))+c_2(1+sin(x))$:
(suppose $c_1=c_2=1$ for simplicity)
\begin{eqnarray*}
(2+cos(x)+sin(x))''+(2+cos(x)+sin(x))-1=0\\
-cos(x)-sin(x)+(2+cos(x)+sin(x))-1\neq 0\\
\end{eqnarray*}
The linear combination of the two solutions is NOT a solution to the ODE. (because the solution is nonhomogeneous)
\end{exmp}

\begin{center}
\noindent\rule{4cm}{0.4pt}
\end{center}


It is also important to remember that in order to apply the superposition, the solutions must be linearly independent from one another!

\begin{center}
\noindent\rule{4cm}{0.4pt}
\end{center}
\begin{exmp}{Linear independence:}\\
Coming back to the first example:
\begin{equation*}
\frac{d^2 y}{dx^2}=y
\end{equation*}
Do the following superposed solutions form the general solution to the ODE?
\begin{equation*}
y=2 e^x+ 4 e^x
\end{equation*}
This is not the general solution to the ODE as the two solutions are not linearly independent!
[Prove!]
\end{exmp}
\begin{center}
\noindent\rule{4cm}{0.4pt}
\end{center}

How can we tell if solutions are linearly independent from each other? We can evaluate the \textbf{Wronski determinant} or also called \textbf{Wronskian}.\\

Given two functions $y_1$ and $y_2$, the Wronskian, $W(y_1,y_2)$ is defined as:
\begin{equation*}
W(y_1,y_2)=\left|\begin{bmatrix}
    y_1 & y_2  \\
    y'_1 & y'_2
  \end{bmatrix}\right| = y_1y'_2 -y'_1y_2
\end{equation*}
(where $\left| \cdot \right|$ is the determinant!)

If $W(y_1,y_2)\neq 0$ the functions are linearly independent!
\begin{center}
\noindent\rule{4cm}{0.4pt}
\end{center}

\begin{exmp}{Linear (in)dependence:}\\
Given the ODE:
\begin{equation*}
y''+ \omega^2 y=0
\end{equation*}
The solutions are $y_1=cos(\omega x)$ and $y_2=sin(\omega x)$. The Wronskian is:
\begin{equation*}
W(y_1,y_2)=\left|\begin{bmatrix}
    cos(\omega x) & sin(\omega x)  \\
    -\omega sin(\omega x) & \omega cos(\omega x)
  \end{bmatrix}\right| = \omega(cos^2(\omega x)+sin^2(\omega x))=\omega \neq 0
\end{equation*}
Therefore the solutions are linearly independent!

\end{exmp}

\begin{center}
\noindent\rule{4cm}{0.4pt}
\end{center}

\begin{testv}{}{}
\subsection{Solution techniques for linear second-order ODEs}
\begin{itemize}
\item Reduction of order
\item Auxiliary equation
\item Method of undetermined coefficients
\item Variation of parameters
\end{itemize}
\end{testv}

%
%\section{Reduction of order}
%\textbf{Basic idea:} Reduce a 2nd order ODE to a 1st order ODE by introducing a new function $V(x)=\frac{dy}{dx}$.\\
%
%Two main cases are examined:
%\begin{itemize}
%\item Equation with "missing terms" can be directly solved (today)
%\item Know one solution, and we want to find a second linearly independent solution (next thursday)
%\end{itemize}
%
%\subsection{Missing terms}
%Recall the standard form of a 2nd order linear ODE:
%\begin{equation*}
%\boxed{\frac{d^2y}{dx^2} +p(x)\frac{dy}{dx}+ q(x)y=0}
%\end{equation*}
%From this standard three cases can be distinguished:
%\begin{enumerate}
%\item missing $p(x)\frac{dy}{dx}$ and $q(x)y$
%\item missing $q(x)y$
%\item missing $p(x)\frac{dy}{dx}$ 
%\end{enumerate}
%
%\subsubsection{General procedure}
%\begin{enumerate}
%\item Introduce $V(x)=\frac{dy}{dx}$
%\item Sub $V(x)$ into the ODE and get a 1st order ODE for $V(x)$
%\item Solve for $V(x)$
%\item Solve for $y$ by integration $y(x)=\int V dx$
%\end{enumerate}
%
%\begin{center}
%\noindent\rule{4cm}{0.4pt}
%\end{center}
%
%\begin{exmp}{Order reduction:}\\
%Solve:
%\begin{equation*}
%\frac{d^2y}{dx^2}=4x
%\end{equation*}
%Solution:
%\begin{enumerate}
%\item Introduce  $V(x)=\frac{dy}{dx}$ and $\frac{V(x)}=\frac{d^2y}{dx^2}$
%\item Sub $V(x)$ into the ODE
%\begin{equation*}
%\frac{dV(x)}{dx}=4x
%\end{equation*}
%
%\item Solve (Separation and integration!)
%\begin{equation*}
%V(x)=\int 4x dx +C = 2x^2 +c_1
%\end{equation*}
%\item Solve for $y$\\
%\begin{equation*}
%y(x)=\int V dx=\int (2x^2 +c_1) dx = \frac{2}{3}x^3 +c_1x +c_2
%\end{equation*}
%
%Our General solution is:
%\begin{equation*}
%y(x)= \frac{2}{3}x^3 +c_1x +c_2
%\end{equation*}
%Note: we have 2 integrative constants (as we have a second-order ODE).
%\end{enumerate}
%\end{exmp}
%
\begin{center}
\noindent\rule{4cm}{0.4pt}
\end{center}

\updateinfo[October 10, 2018]